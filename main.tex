\documentclass{report}   %Documento tipo reporte
\usepackage[spanish]{babel}    %Paquete de Idioma
\usepackage{hyperref}   %Paquete hipervínculo

\begin{document}


\begin{titlepage}    %Portada
	\centering
	{\scshape\LARGE Universidad de Antioquia \par}
	\vspace{1cm}
	{\scshape\Large Proyecto de Investigación.  \par}
	\vspace{1.5cm}
	{\huge\bfseries La dialéctica en el desarrollo de nuevas tecnologías\par}
	\vspace{2cm}
	{\Large\itshape Valentina Botero Vivas \par}
	\vfill
    Supervisado por\par
	Augusto Salazar Jimenez\textsc{}

	\vfill

	{\large 26 de Marzo del 2020 \par}
\end{titlepage}


\section{Introducción}

El recorrido descriptivo del Mago y matemático Nelo Maestre y la periodista y divulgadora científica Ágata Timón en su ensayo: “Así terminó el sueño de las matemáticas infalibles”, nos ubican en el centro de un problema o asunto propio de la historia de las ideas y muy específicamente las condiciones del saber matemático como apodíctico, axiomático y sin contemplar la posibilidad del error, la infalibilidad o la duda misma.  

Gracias a la revolución industrial se modificaron e influenciaron aspectos de la vida cotidiana en la sociedad. Pero a finales del siglo XIX el mundo fue testigo de una revolución que permite extraer todo el potencial de estos cambios con una herramienta invisible pero eficaz: La matemática de manera abstracta.
En ese siglo aparecen numerosas teorías y se demuestran conjeturas expuestas anteriormente. La matemática adquiere una importancia nunca antes vista y las aplicaciones se desarrollan en amplios dominios y surge la idea de desplazar las nociones a infinitamente pequeñas o infinitamente grandes.

¿Pero a qué se refieren con infinitamente? ¿Existe la posibilidad de contar infinitos?



\section{Desarrollo}

 El matemático Georg Cantor nacido en Rusia, fue la primera persona que pudo formalizar un concepto, que durante miles de años fue incomprendido por el ser humano por el hecho de ser absolutamente inalcanzable… El infinito. Afirmaba a diferencia de sus colegas de dicha época, que:
 
 \begin{quote}
      ``El miedo al infinito es una forma de miopía que destruye la posibilidad de ver el infinito real, a pesar de que en su forma más elevada nos ha creado y sostenido." 

 \end{quote}

 Cantor reveló  gracias a su "Teoría de conjuntos” que el infinito en sí mismo es un número y además que hay infinitos de distintos tamaños.
Esto lleva a la comunidad científica a percatarse de la excesiva confianza concedida a la intuición hasta el momento, en consecuencia a esto son severamente afectados pilares de la ciencia como lo son la verdad y la demostración, elemento que sacude los cimientos de los fundamentos, provocando una verdadera renovación en la forma de ver lo matemático hasta entonces, he ahí la crisis de los fundamentos, una crisis que da lugar al desarrollo de la  ciencia y sus consecuentes avances tecnológicos. 

La ciencia, las disciplinas o soportes de ellas, como lo es la matemática, no pueden dar por cancelada la discusión y tener una versión definitiva e intocable sobre sus conceptos o fundamentos, como lo pretendían los formalistas liderados por David Hilbert, apoyados en una triada sin dar lugar a discusión: La consistencia (sin contradicciones), la finitud como garantía de convergencia en un rango y la completes, cada afirmación que se haga debe ser susceptible de demostrar su falsedad o veracidad. 

Pero el discurso científico no se puede dar por completado y menos clausurado en su esencia de desarrollo y progresividad, las épocas cambian sus paradigmas, lo que antes era tenido por verdadero o indiscutible puede ser objeto de refutación, tal es lo que hace Kurt Gödel con su “Teorema de la Incompletitud”, que reconoce y fundamenta algo que debía ser  por lo menos evidente, el saber tiene unos elementos contradictorios, con posibilidad en el error y en consecuencia la posibilidad de ser verdadero o falso. 

Gracias a este desarrollo de reconocer lo diverso y válido de la contradicción en el desarrollo científico, Alan Turing pudo instrumentalizar este concepto y mostró además que no se puede saber de antemano cual problema cumplirá lo que hace que a veces se entre en el campo de la incertidumbre.



\section{Conclusión}

Solo gracias a la posibilidad de dudar ante lo evidente, cambiar paradigmas, hizo posible el desarrollo de una máquina y una metodología que se considera un salto cualitativo en la historia humana, como lo fue la invención del fuego, o la rueda… La computación, un elemento esencial del actual desarrollo tecnológico, científico, y una variación de lo cotidiano y la noción de realidad que contemporáneamente hemos construido.

\section{Glosario}
    \subsection{Apodíctico: }
       \paragraph{Expresa o encierra una verdad concluyente que no deja lugar a duda.}
    \subsection{Axioma: }
        \paragraph{ Proposición o enunciado tan evidente que se considera que no requiere demostración.}
    \subsection{Dialéctica: }
        \paragraph{Técnica retórica de dialogar y discutir para descubrir la verdad mediante la exposición y confrontación de razonamientos y argumentaciones contrarios entre sí.}
    \subsection{Doxa:}
        \paragraph{ Opinión o conocimiento obtenido por la experiencia.}
    \subsection{Episteme:}
        \paragraph{ Es el verdadero conocimiento, de la verdadera realidad.}
    \subsection{Subjetivo:}
        \paragraph{ Que se basa en los sentimientos o creencias de cada persona.}
    \subsection{Objetivo:}
        \paragraph{ Basado en hechos.}
    \subsection{Paradigmas:}
        \paragraph{ Refleja algo en específico que funciona como un ejemplo a seguir. También se utiliza para señalar aquello que funciona y es tomado como modelo.}
        
\section{Cibergrafía}
    \subsection{Nelo Maestre y Ágata Timón.(2018,Semptiembre 20).``Así terminó el sueño de las matemáticas infalibles (y de paso, nació la computación moderna)".[Online].\href{https://www.bbvaopenmind.com/ciencia/matematicas/asi-termino-el-sueno-de-las-matematicas-infalibles/?utm_source=materia&utm_medium=facebook&tipo=elabora&cid=soc:afl:fb:----materia:--:::::::sitlnk:materia:&fbclid=IwAR2oBVdboWbxmrIAqVbiA2hZrBTjcvLSl6RQ5Fh5UQX2ZvvybK8MlpGi3gM}{Disponible: URL.}}
    \subsection{María Estela Raffino.(2020, Marzo 26).``Computación".[Online].
    \href{https://concepto.de/computacion/}{Disponible: URL.}}
    \subsection{[3]Fouce, J.M.(2015,Enero 3).``La Filosofía en webdianoia".[Online]
    \href{http://www.webdianoia.com/index.html}{Disponible: URL.}}
    \subsection{Marcus du Sautoy.(2018,Semptiembre 2).``Georg Cantor, el matemático que descubrió que hay muchos infinitos y no todos son del mismo tamaño".[Online].
    \href{https://www.bbc.com/mundo/noticias-45300219}{Disponible: URL.}}
    \subsection{WordReference.(2005).``Diccionario sinónimos y antónimos."[Online].
    \href{https://www.wordreference.com/}{Disponible: URL.}}

\end{document}
