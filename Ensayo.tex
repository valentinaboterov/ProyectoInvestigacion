\documentclass{report}
\usepackage[utf8]{inputenc}

\title{La dialéctica en el desarrollo de nuevas tecnologías.}
\author{Valentina Botero Vivas }
\date{23 Marzo 2020}

\begin{document}

\maketitle

El recorrido descriptivo del Mago y matemático Nelo Maestre y la periodista y divulgadora científica Ágata Timón en su ensayo: “Así terminó el sueño de las matemáticas infalibles”, nos ubican en el centro de un problema o asunto propio de la historia de las ideas y muy específicamente las condiciones de posibilidad del saber matemático como apodíctico, axiomático y sin contemplar la posibilidad del error, la infalibilidad o la duda misma.  

La ciencia, las disciplinas o soportes de ellas, como lo es la matemática, no pueden dar por cancelada la discusión y tener una versión definitiva e intocable sobre sus conceptos o fundamentos, como lo pretenden los formalistas liderados por David Hilbert, apoyados en una triada sin dar lugar a discusión: La consistencia (sin contradicciones), la finitud como garantía de convergencia en un rango (un deber ser para las cosas) y la completes, cada afirmación que se haga debe ser susceptible de demostrar su falsedad o veracidad. 

Pero el discurso científico no se puede dar por completado y menos clausurada en su esencia de desarrollo y progresividad, las épocas cambian sus paradigmas, lo que antes era tenido por verdadero o indiscutible puede ser objeto de refutación, tal es lo que hace Kurt Gödel con su “Teorema de la Incompletitud”, que reconoce y fundamenta algo que debía ser  por lo menos evidente, el saber tiene unos elementos contradictorios, con posibilidad en el error y en consecuencia la posibilidad de ser verdadero o falso. 

Esto lleva a la comunidad científica a percatarse de la excesiva confianza concedida a la intuición hasta el momento, en consecuencia a esto son severamente afectados pilares de la ciencia como lo son la verdad y la demostración, elemento que sacude los cimientos de los fundamentos, provocando una verdadera renovación en la forma de ver lo matemático hasta entonces, he ahí la crisis de los fundamentos, una crisis que da lugar al desarrollo de la  ciencia y sus consecuentes avances tecnológicos. 

Gracias a este desarrollo de reconocer lo diverso y válido de la contradicción en el desarrollo científico, Alan Turing pudo instrumentalizar este concepto y mostró además que no se puede saber de antemano cual problema cumplirá lo que hace que a veces se entre en el campo de la incertidumbre.

Solo gracias a la posibilidad de dudar ante lo evidente, cambiar paradigmas, hizo posible el desarrollo de una máquina y una metodología que se considera un salto cualitativo en la historia humana, como lo fue la invención del fuego, del cero, de la rueda… La computación, un elemento esencial del actual desarrollo tecnológico, científico, y una variación de lo cotidiano y la noción de realidad que contemporáneamente hemos construido.

A continuación, un Glosario de términos que tiene algún vínculo con la naturaleza del texto y ayudan a esclarecerlo.

\section{Glosario:}
Apodíctico:  Expresa o encierra una verdad concluyente que no deja lugar a duda. 

Axioma: Proposición o enunciado tan evidente que se considera que no requiere demostración.

Dialéctica: Técnica retórica de dialogar y discutir para descubrir la verdad mediante la exposición y confrontación de razonamientos y argumentaciones contrarios entre sí.

Doxa: Opinión o conocimiento obtenido por la experiencia. 

Episteme: Es el verdadero conocimiento, de la verdadera realidad.

Subjetivo: que se basa en los sentimientos o creencias de cada persona.

Objetivo: Basado en hechos.

Paradigmas: Refleja algo en específico que funciona como un ejemplo a seguir. También se utiliza para señalar aquello que funciona y es tomado como modelo.

\section{Cibergrafía:}

https://www.bbvaopenmind.com/ciencia/matematicas/asi-termino-el-sueno-de-las-matematicas-infalibles/?utm_source=materia&utm_medium=facebook&tipo=elabora&cid=soc:afl:fb:----materia:--:::::::sitlnk:materia:&fbclid=IwAR2oBVdboWbxmrIAqVbiA2hZrBTjcvLSl6RQ5Fh5UQX2ZvvybK8MlpGi3gM

\end{document}